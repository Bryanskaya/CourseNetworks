
\begin{lstlisting}[caption = {Класс TorrentClient}]
class PeerConnection:
	remote_id = None
	writer: asyncio.StreamWriter = None
	reader: asyncio.StreamReader = None
	
		def __init__(self, id: int, queue: Queue, info_hash: bytes, peer_id: str,
		piece_manager: PieceManager, block_cb: Callable[[str, int, int, bytes], None]):
		self.state = PeerState()
		self.peer_state = OtherPeerState()
		
		self.id = id
		self.queue = queue
		self.info_hash = info_hash
		self.peer_id = peer_id
		self.piece_manager = piece_manager
		self.block_cb = block_cb  # Callback function. It is called when block is received from the remote peer.
		self.future = asyncio.ensure_future(self._start())
	
	async def _start(self):
		while not self.state.is_stopped:
			ip, port = await self.queue.get()
			self._info("assigned peer, ip = {}".format(ip))
			
			try:
				self.reader, self.writer = await asyncio.open_connection(ip, port)
				self._info("connection was opened, ip = " + ip)
				
				buf = await self._do_handshake()
				await self._send_interested()
				
				async for msg in PeerStreamIterator(self.reader, buf):
					self._info("received {}".format(str(msg)))
					if self.state.is_stopped:
						break
				
					await self._process_response(msg)
					await self._create_message()
	
			except concurrent.futures._base.CancelledError as exp:
				self._exep(exp)
				await self.cancel()
				raise exp
	
			except Exception as exp:
				self._exep(exp)
				await self.cancel()
				raise exp
			self._info('Out of loop')
			await self.cancel()
	
	async def _process_response(self, msg: PeerMessage):
		if type(msg) is BitFieldMsg:
			msg: BitFieldMsg
			self.piece_manager.add_peer(self.remote_id, msg.bitfield)
		elif type(msg) is InterestedMsg:
			self.peer_state.interest()
		elif type(msg) is NotInterestedMsg:
			self.peer_state.uninterest()
		elif type(msg) is ChokeMsg:
			self.state.choke()
		elif type(msg) is UnchokeMsg:
			self.state.unchoke()
		elif type(msg) is HaveMsg:
			msg: HaveMsg
			self.piece_manager.update_peer(self.remote_id, msg.index)
		elif type(msg) is KeepAliveMsg:
			pass
		elif type(msg) is PieceMsg:
			msg: PieceMsg
			self.state.stop_pending()
			self.block_cb(self.remote_id, msg.index, msg.begin, msg.block)
		elif type(msg) is RequestMsg:
			pass
		elif type(msg) is CancelMsg:
			pass
		
	async def _create_message(self):
		if (not self.state.is_choked) and self.state.is_interested and (not self.state.is_pending):
			self.state.start_pending()
			success = await self._request_piece()
			if not success:
				self.state.stop_pending()
	
	async def cancel(self):
		self._info('closing peer {ip}'.format(ip=self.remote_id))
		self.remote_id = None
		if not self.future.done():
			pass 
		if self.writer:
			self.writer.close()
			await self.writer.wait_closed()
		
		self.queue.task_done()
	
	def stop(self) -> None:
		self.state.stop()
		if not self.future.done():
			self.future.cancel()
	
	async def _do_handshake(self) -> bytes:
		self.writer.write(HandshakeMsg(self.info_hash, self.peer_id).encode())
		await self.writer.drain()
		
		buf = b''
		tries = 0
		while len(buf) < HandshakeMsg.length and tries < 10:
			tries += 1
			buf += await self.reader.read(PeerStreamIterator.CHUNK_SIZE)
		
		response = HandshakeMsg.decode(buf)
		if response is None:
			raise Exception()
		elif response.info_hash != self.info_hash:
			raise Exception()
		
		self.remote_id = response.peer_id
		self._debug('handshake with peer {} are successful'.format(self.remote_id))
		
		return buf[HandshakeMsg.length:]
	
	async def _send_interested(self) -> None:
		msg = InterestedMsg()
		self.writer.write(msg.encode())
		await self.writer.drain()
		self._debug('sending interested msg to {}'.format(self.remote_id))
	
	async def _request_piece(self) -> bool:
		block = self.piece_manager.next_request(self.remote_id)
		if block is None:
			return False
	
		msg = RequestMsg(block.piece, block.offset, block.length).encode()
		
		self._debug('Requesting block {} of piece {} from {}'.
	format(block.offset, block.piece, self.remote_id))
	
		self.writer.write(msg)
		await self.writer.drain()
		return True
	
\end{lstlisting}