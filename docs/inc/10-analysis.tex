\section{Аналитическая часть}

\subsection{Постановка задачи}
Результатом работы должна стать программа для загрузки файлов по протоколу Bittorrent, удовлетворяющая следующим требованиям:
\begin{itemize}
	\item поддерживать файлы расширения torrent;
	
	\item поддерживать функцию загрузки данных как от сервера, так и от других клиентов;
	
	\item обладать графическим интерфейсом для удобства выполнения действий и просмотра текущей информации по состоянию загрузки.
	
\end{itemize}

Первостепенной задачей для дальнейшей разработки является изучения устройства выбранного протокола. \newline

\subsection{Принцип работы протокола}
Bittorrent -- P2P протокол для кооперативного обмена файлами через интернет. 

В данном протоколе выделены две роли:
\begin{enumerate}
	\item \textbf{пир} (клиент) хранит файлы и производит обмен их частями с другими пирами;
	
	\item \textbf{трекер} (сервер) хранит таблицу файлов и список пиров, имеющих данный файл в распоряжении.
\end{enumerate}

Пир, желающий получить файл должен обладать \textbf{.torrent файлом}, с помощью которого он может обратиться к серверу. Сервер предоставляет адреса клиентов, обладающих запрашиваемыми файлами после чего начинается их загрузка. Передача осуществляется частями (\textbf{pieces}), каждый torrent-клиент, скачивая эти части, в то же время отдаёт их другим клиентам, что снижает нагрузку на каждого отдельного клиента. \newline

\subsection{Структура .torrent файла}
Как было отмечено выше, первым шагом в начале загрузки является получение и парсинг файла специального расширения .torrent.

Для кодирования данных в .torrent-файлах используется формат Bencode. Само содержимое -- ассоциативный массив с полями:
\begin{itemize}
	\item \textbf{info} -- вложенный ассоциативный массив который собственно и описывает файлы, которые передаёт торрент;
	
	\item \textbf{announce} -- URL трекера;
	
	\item \textbf{announce-list} -- список трекеров, если их несколько, в Bencode-виде — список списков;
	
	\item \textbf{creation date} -- дата создания;
	
	\item \textbf{comment} -- текстовое описание торрента;
	
	\item \textbf{created by} -- автор торрента. \\
\end{itemize}

info и announce являются обязательными полями, всё остальное — опционально. Первый в свою очередь состоит из:
\begin{itemize}
	\item \textbf{piece length} -- размер одного куска;
	
	\item \textbf{pieces} -- конкатенация SHA1-хешей каждого куска (каждый хэш - 20 символов);
	
	\item \textbf{name} -- имя файла (если файл один);
	
	\item \textbf{length} -- содержит длину файла (если файл один);
	
	\item \textbf{files} -- если файлов несколько, то содержит список ассоциативных массивов (с указанием length и path). \\
\end{itemize}

Данная информация используется на всём протяжении загрузки файла и его последующей раздаче. \newline

\subsection{Взаимодействие клиента и сервера}





\subsection{Структура сообщений}




\subsection{Взаимодействие клиентов}
