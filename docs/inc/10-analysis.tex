\section{Аналитическая часть}

\subsection{Постановка задачи}
Результатом работы должна стать программа для загрузки файлов по протоколу Bittorrent, удовлетворяющая следующим требованиям:
\begin{itemize}
	\item поддерживать файлы расширения torrent;
	
	\item поддерживать функцию загрузки данных как от сервера, так и от других клиентов;
	
	\item обладать графическим интерфейсом для удобства выполнения действий и просмотра текущей информации по состоянию загрузки.
	
\end{itemize}

Первостепенной задачей для дальнейшей разработки является изучения устройства выбранного протокола. \newline

\subsection{Принцип работы протокола}
Bittorrent -- специальный протокол для скачивания файлов и их распространения через интернет. 

В данном протоколе выделены две роли:
\begin{enumerate}
	\item \textbf{пир} (клиент) хранит файлы и производит обмен их частями с другими пирами;
	
	\item \textbf{трекер} (сервер) хранит таблицу файлов и список пиров, имеющих данный файл в распоряжении.
\end{enumerate}

Пир, желающий получить файл должен обладать \textbf{.torrent файлом}, с помощью которого он может обратиться к серверу. Сервер предоставляет адреса клиентов, обладающих запрашиваемыми файлами после чего начинается их загрузка. Передача осуществляется частями (\textbf{pieces}), каждый torrent-клиент, скачивая эти части, в то же время отдаёт их другим клиентам, что снижает нагрузку на каждого отдельного клиента. \newline

\subsection{Структура .torrent файла}
Как было отмечено выше, первым шагом в начале загрузки является получение и парсинг файла специального расширения .torrent. Рассмотрим хранимый в нём набор информации и её формат.

\subsection{Структура сообщений}

\subsection{Взаимодействие клиента и сервера}

\subsection{Взаимодействие клиентов}
