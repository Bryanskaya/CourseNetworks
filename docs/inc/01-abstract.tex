\section*{ВВЕДЕНИЕ}
\addcontentsline{toc}{section}{ВВЕДЕНИЕ}

За последние время существенно возросли объёмы информации, передаваемой по сети Интернет. Очевидно, что подобная тенденция сохранится и в будущем -- будет расти число пользователей и объём потребляемого ими трафика.

В подобных условиях актуальным является вопрос производительности серверов. Ввиду описанных выше факторов нагрузка на них будет постоянно расти, что будет вынуждать их владельцев производить их обновление и расширение или снижение скорости обмена информацией с клиентами.

Последнее является чувствительным для загрузки файлов больших объёмов. Решением в таком случае может быть кооперативный обмен файлами. Наиболее популярным протоколом для этой технологии является Bittorrent.

\textbf{Целью} данной работы является разработка Bittorrent клиента.

Для достижения поставленной цели необходимо решить следующие \textbf{задачи}:
\begin{enumerate}
	\item изучить структуру и принцип работы протокола;
	
	\item разработать алгоритм взаимодействия с сервером и клиентами;
	
	\item реализовать программу для загрузки файлов на основе протокола Bittorrent.
\end{enumerate}