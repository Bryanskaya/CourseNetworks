\usepackage[russian]{babel}
\usepackage[utf8]{inputenc}
\usepackage[T2A]{fontenc}
\usepackage{amsfonts}
\usepackage{amsmath}
\usepackage[
left 	= 	30	mm,
right 	=	15	mm,
top 	=	20	mm,
bottom 	=	20	mm,
]{geometry}

\usepackage{indentfirst} % Красная строка

\setlength{\parindent}{1.25 cm}

\usepackage[toc,page]{appendix}

\renewcommand{\baselinestretch}{1.5}

\usepackage{titlesec}

\titleformat{\section}
{\normalfont\fontsize{14}{14}\bfseries}{\thesection}{1em}{}

\titleformat{\subsection}
{\normalfont\fontsize{14}{14}\bfseries}{\thesubsection}{1em}{}

\usepackage[intoc]{nomencl}
\renewcommand{\nomname}{Обозначения и сокращения}
\makenomenclature

%%% Математика

% Шрифты для математики
\usepackage{amsmath}
\usepackage{amsfonts}
\usepackage{amssymb}
\usepackage{cancel}
\usepackage{mathrsfs}
\usepackage{mathtools}
\usepackage{upgreek}
\usepackage{xfrac}


%%% Иллюстрации
\usepackage{graphicx}
\usepackage{subcaption}
\usepackage{wrapfig}
\usepackage[export]{adjustbox}
%\graphicspath{{./img/}}

\makeatletter % список литературы
\def\@biblabel#1{#1. }
\makeatother

% Графики
\usepackage{pgfplots}
\pgfplotsset{compat=1.3}
\usepgfplotslibrary{patchplots}
%\usepackage{patchplots}
\pgfplotsset{	width	=	14	cm,
	x label style={
		font = {\small\sffamily},
		yshift = 1mm
	},
	tick label style={
		font = {\scriptsize},
	},
	y label style={
		font = {\small\sffamily},
		yshift = -1mm,
		at={(ticklabel cs:0.5)},
		%      					rotate=90,
		anchor=near ticklabel
	},
	every tick/.style	=	{
		black, 
		line width 	= 	.5	pt
	},
	axis line style 	= 	{
		line width 	= 	.5	pt
	},
	grid style	=	{
		gray,
		dotted
	},
	minor x tick num = 1,
	minor y tick num = 1,
	no markers,
	grid = major,
	every axis/.append style	=	{
		line width	=	.7	pt
	}
}

%Подписи
\usepackage		[margin		= 10	pt,
%					font		= footnotesize, 
%labelfont	= bf, 
labelsep	= endash, 
%labelfont	= bf,
%					textfont	= sl,
margin		= 0 	pt,  
aboveskip 	= 4		pt, 
belowskip 	= -6	pt,
figurename= Рисунок] {caption}
\usepackage		[margin		= 10	pt,
font		= footnotesize, 
labelfont	= bf, 
labelsep	= endash, 
labelfont	= bf,
textfont	= sl,
margin		= 0 	pt,  
aboveskip 	= 4		pt, 
belowskip 	= 6	pt]	{subcaption}

\makeatletter
%\newcounter{figure}[section]
%\newcounter{table}[section]
\renewcommand{\thefigure}{\thesection.\@arabic\c@figure}
\renewcommand{\thetable}{\thesection.\@arabic\c@table}
\makeatother


%%% Insert pdf pages
\usepackage[final]{pdfpages}


%%% Color highlight
\usepackage{xcolor}

% Ссылки внутри текста
\usepackage{hyperref}

% Настройка листингов
\usepackage{listings}
\lstset{
	language = sql,
	extendedchars=\true,
	keepspaces=true,
	basicstyle=\scriptsize\sffamily,
	showstringspaces=\false,
	numbers=left,
	stepnumber=1,
	numbersep=5pt,
	frame=single,
	tabsize=2,
	captionpos=t,
	breaklines=true,
	breakatwhitespace=false,
	escapeinside={\#*}{*)}
}